
\documentclass[journal,12pt,twocolumn]{IEEEtran}

\usepackage{graphicx}
\usepackage{setspace}
\usepackage{gensymb}
\singlespacing
\usepackage[cmex10]{amsmath}
\usepackage{amssymb}
\usepackage{xurl}
\usepackage{tabularx}
\usepackage{amsthm}
\usepackage{comment}
\usepackage{mathrsfs}
\usepackage{txfonts}
\usepackage{stfloats}
\usepackage{bm}
\usepackage{cite}
\usepackage{cases}
\usepackage{subfig}
\usepackage{arydshln}
\usepackage{longtable}
\usepackage{multirow}

\usepackage{tikz}
\usetikzlibrary{positioning,fit,calc}
\tikzstyle{block} = [draw=black, thick, text width=3cm, minimum height=2cm, align=center]  
\tikzstyle{arrow} = [thick,->,>=stealth]

\usepackage{enumitem}
\usepackage{mathtools}
\usepackage{steinmetz}
\usepackage{tikz}
\usepackage{circuitikz}
\usepackage{verbatim}
\usepackage{tfrupee}
\usepackage[breaklinks=true]{hyperref}
\usepackage{graphicx}
\usepackage{tkz-euclide}
\usetikzlibrary{automata, positioning}
\usetikzlibrary{calc,math}
\usepackage{listings}
    \usepackage{color}                                            %%
    \usepackage{array}                                            %%
    \usepackage{longtable}                                        %%
    \usepackage{calc}                                             %%
    \usepackage{multirow}                                         %%
    \usepackage{hhline}                                           %%
    \usepackage{ifthen}                                           %%
    \usepackage{lscape}     
\usepackage{multicol}
\usepackage{chngcntr}
\usepackage{blkarray}

\DeclareMathOperator*{\Res}{Res}

\renewcommand\thesection{\arabic{section}}
\renewcommand\thesubsection{\thesection.\arabic{subsection}}
\renewcommand\thesubsubsection{\thesubsection.\arabic{subsubsection}}

\renewcommand\thesectiondis{\arabic{section}}
\renewcommand\thesubsectiondis{\thesectiondis.\arabic{subsection}}
\renewcommand\thesubsubsectiondis{\thesubsectiondis.\arabic{subsubsection}}


\hyphenation{op-tical net-works semi-conduc-tor}
\def\inputGnumericTable{}                                 %%

\lstset{
%language=C,
frame=single, 
breaklines=true,
columns=fullflexible
}
\begin{document}


\newtheorem{theorem}{Theorem}[section]
\newtheorem{problem}{Problem}
\newtheorem{proposition}{Proposition}[section]
\newtheorem{lemma}{Lemma}[section]
\newtheorem{corollary}[theorem]{Corollary}
\newtheorem{example}{Example}[section]
\newtheorem{definition}[problem]{Definition}

\newcommand{\BEQA}{\begin{eqnarray}}
\newcommand{\EEQA}{\end{eqnarray}}
\newcommand{\define}{\stackrel{\triangle}{=}}
\bibliographystyle{IEEEtran}
\raggedbottom
\setlength{\parindent}{0pt}
\providecommand{\mbf}{\mathbf}
\providecommand{\pr}[1]{\ensuremath{\Pr\left(#1\right)}}
\providecommand{\qfunc}[1]{\ensuremath{Q\left(#1\right)}}
\providecommand{\sbrak}[1]{\ensuremath{{}\left[#1\right]}}
\providecommand{\lsbrak}[1]{\ensuremath{{}\left[#1\right.}}
\providecommand{\rsbrak}[1]{\ensuremath{{}\left.#1\right]}}
\providecommand{\brak}[1]{\ensuremath{\left(#1\right)}}
\providecommand{\lbrak}[1]{\ensuremath{\left(#1\right.}}
\providecommand{\rbrak}[1]{\ensuremath{\left.#1\right)}}
\providecommand{\cbrak}[1]{\ensuremath{\left\{#1\right\}}}
\providecommand{\lcbrak}[1]{\ensuremath{\left\{#1\right.}}
\providecommand{\rcbrak}[1]{\ensuremath{\left.#1\right\}}}
\theoremstyle{remark}
\newtheorem{rem}{Remark}
\newcommand{\sgn}{\mathop{\mathrm{sgn}}}
\providecommand{\abs}[1]{\vert#1\vert}
\providecommand{\res}[1]{\Res\displaylimits_{#1}} 
\providecommand{\norm}[1]{\lVert#1\rVert}
%\providecommand{\norm}[1]{\lVert#1\rVert}
\providecommand{\mtx}[1]{\mathbf{#1}}
\providecommand{\mean}[1]{E[ #1 ]}
\providecommand{\fourier}{\overset{\mathcal{F}}{ \rightleftharpoons}}
%\providecommand{\hilbert}{\overset{\mathcal{H}}{ \rightleftharpoons}}
\providecommand{\system}{\overset{\mathcal{H}}{ \longleftrightarrow}}
	%\newcommand{\solution}[2]{\textbf{Solution:}{#1}}
\newcommand{\solution}{\noindent \textbf{Solution: }}
\newcommand{\cosec}{\,\text{cosec}\,}
\providecommand{\dec}[2]{\ensuremath{\overset{#1}{\underset{#2}{\gtrless}}}}
\newcommand{\myvec}[1]{\ensuremath{\begin{pmatrix}#1\end{pmatrix}}}
\newcommand{\mydet}[1]{\ensuremath{\begin{vmatrix}#1\end{vmatrix}}}
\newcommand*{\permcomb}[4][0mu]{{{}^{#3}\mkern#1#2_{#4}}}
\newcommand*{\perm}[1][-3mu]{\permcomb[#1]{P}}
\newcommand*{\comb}[1][-1mu]{\permcomb[#1]{C}}
\numberwithin{equation}{subsection}
\makeatletter
\@addtoreset{figure}{problem}
\makeatother
\let\StandardTheFigure\thefigure
\let\vec\mathbf
\renewcommand{\thefigure}{\theproblem}
\def\putbox#1#2#3{\makebox[0in][l]{\makebox[#1][l]{}\raisebox{\baselineskip}[0in][0in]{\raisebox{#2}[0in][0in]{#3}}}}
     \def\rightbox#1{\makebox[0in][r]{#1}}
     \def\centbox#1{\makebox[0in]{#1}}
     \def\topbox#1{\raisebox{-\baselineskip}[0in][0in]{#1}}
     \def\midbox#1{\raisebox{-0.5\baselineskip}[0in][0in]{#1}}
\vspace{3cm}
\title{GATE ASSIGNMENT 1}
\author{HARITHA R\\ AI20BTECH11010}
\maketitle
\newpage
\bigskip
\renewcommand{\thefigure}{\arabic{figure}}
\renewcommand{\thetable}{\arabic{table}}
Download all python codes from
\begin{lstlisting}
https://github.com/harithar1234/EE3900-Haritha/blob/main/gateassignment1/gateassignment1.py
\end{lstlisting}
\section*{QUESTION}
\begin{large}
\textbf{EC-2019 Q.29}\\
It is desired to find a three-tap causal filter which gives zero signal as an output to an input of the form
\[x[n]  = c_1 \exp{(\frac{-j\pi n}{2})} +c_2 \exp{(\frac{j\pi n}{2})}\]
where $c_1$ and $c_2$ are arbitrary real numbers. The desired three-tap filter is given by
\[h[0]=1, h[1]=a ,h[2]=b\]
and
\[h[n]=0 \text{ for } n<0 \text{ or } n>2.\]
What are the values of the filter taps a and b if the output is y[n] = 0 for all n, when x[n] is as given above?\\
\begin{center}
\begin{tikzpicture}  
  \node[block] (a) {n=0\\ \downarrow \\h[n]=\{1,a,b\}};  
  \draw[<-] (4,0) -- node[above]{y[n]=0}
  (a);
  \draw[<-] (a) -- node[above]{x[n]} (-3,0);
\end{tikzpicture}
\end{center}
(A)a=1,b=1\\
(B)a=0,b=-1\\
(C)a=-1,b=1\\
(D)a=0,b=1\\


\section*{SOLUTION}

given:
y[n] = 0 for all n.
\begin{align}
x[n]  = c_1 e^{\frac{-j\pi n}{2}} +c_2 e^{\frac{j\pi n}{2}}
\end{align}
h[0] = 1, h[1] = a, h[2] = b and\\
h[n] = 0 for n $<$ 0 or n $>$ 2.\\
answer:
A discrete-time LTI system with impulse response h[n] has a transfer function $ H(z)=\sum_{n=-\infty}^{\infty}h(n)z^{-n} $\\
An N-tap FIR filter with coefficients h[k] in general has Z-transform  given by $ \sum_{n=0}^{N-1}h(n)z^{-n}  $\\

for the 3-tap FIR filter, the  Z-transform is:
\begin{align}
    H(z)= \sum_{n=0}^{2}h(n)z^{-n} 
\end{align}
The 3-tap filter's frequency response $H(e^{jw})$ obtained by substituting $z=e^{jw}$ is:
\begin{align}
    H(e^{jw})= \sum_{n=0}^{2}h(n)e^{-jnw}\\
    H(e^{jw})=h(0)e^{-0jw}+h(1)e^{-1jw}+h(2)e^{-2jw}\\
    H(e^{jw})=1+ae^{-jw}+be^{-2jw}
\end{align}  
output of discrete-time LTI system y[n] for a given input sequence x[n] is given by the convolution sum :
$y[n]=\sum_{k=-\infty}^{\infty}h[k]x[n-k]$\\


output of discrete-time LTI system with exponential input is $y[n]= \left| {H\left( {{e^{j{\rm{\omega }}}}} \right)} \right|x\left( n \right){e^{\left( {j{\rm{\omega }} + {\rm{\Phi }}} \right)}}$\\

at $w = \frac{-\pi}{2}$
\begin{align}
   H\left( {{e^{-j\frac{\pi }{2}}}} \right) = 1 + a{e^{- j\left( {\frac{-\pi }{2}} \right)}} + b{e^{ - j2\left( {-\pi /2} \right)}}\\
   H\left( {{e^{j\frac{\pi }{2}}}} \right)=(1 - b) +ja
\end{align}
at $w = \frac{\pi}{2}$
\begin{align}
   H\left( {{e^{j\frac{\pi }{2}}}} \right) = 1 + a{e^{ - j\left( {\frac{\pi }{2}} \right)}} + b{e^{ - j2\left( {\pi /2} \right)}}\\
   H\left( {{e^{j\frac{\pi }{2}}}} \right)=(1 - b) - ja\\
   \left| {H\left( {{e^{j\frac{\pi }{2}}}} \right)} \right| = \left| {H\left( {{e^{ - j\frac{\pi }{2}}}} \right)} \right|=\sqrt {{{\left( {1 - b} \right)}^2} + {a^2}} 
\end{align}

Expression of output y[n]:
\begin{align}
  y\left( n \right) = {\left[ {{{\left( {1 - b} \right)}^2} + {a^2}} \right]^{1/2}}\left[ {{c_1}{e^{ - j\;\left( \frac{{n\pi }}{2}  + \Phi_1 \right)}} + {c_2}{e^{j\;\left( \frac{\pi }{2}n + \Phi_2 \right)}}} \right] 
\end{align}
so y[n]=0 always, if $\left| {H\left( {{e^{j{\rm{\omega }}}}} \right)} \right|=0$
\begin{align}
  \sqrt {{{\left( {1 - b} \right)}^2} + {a^2}} =0
\end{align}

from the options\\
b = 1\\
a = 0\\
\textbf{option D}
\end{large}
\end{document}